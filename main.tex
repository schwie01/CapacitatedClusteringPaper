%%%%%%%% ICML 2022 EXAMPLE LATEX SUBMISSION FILE %%%%%%%%%%%%%%%%%

\documentclass[nohyperref]{article}

% Recommended, but optional, packages for figures and better typesetting:
\usepackage{microtype}
\usepackage{icml2022}
\usepackage{graphicx}
\usepackage{subfigure}
\usepackage{booktabs} % for professional tables
\usepackage{graphicx}
\usepackage{epstopdf}
\usepackage{caption}

\usepackage{todonotes}
\usepackage{amsmath,amsthm,amssymb}
\usepackage{xspace}
\usepackage{url}
\usepackage{dsfont}

% hyperref makes hyperlinks in the resulting PDF.
% If your build breaks (sometimes temporarily if a hyperlink spans a page)
% please comment out the following usepackage line and replace
% \usepackage{icml2022} with \usepackage[nohyperref]{icml2022} above.


% Attempt to make hyperref and algorithmic work together better:
\newcommand{\theHalgorithm}{\arabic{algorithm}}

% Use the following line for the initial blind version submitted for review:

% If accepted, instead use the following line for the camera-ready submission:
% \usepackage[accepted]{icml2022}

% For theorems and such
\usepackage{amsmath}
\usepackage{amssymb}
\usepackage{mathtools}
\usepackage{amsthm}

%For tables
\usepackage{multirow}
\usepackage{wrapfig}



% if you use cleveref..
\usepackage[capitalize,noabbrev]{cleveref}

%%%%%%%%%%%%%%%%%%%%%%%%%%%%%%%%
% THEOREMS
%%%%%%%%%%%%%%%%%%%%%%%%%%%%%%%%
\theoremstyle{plain}
\newtheorem{theorem}{Theorem}[section]
\newtheorem{proposition}[theorem]{Proposition}
\newtheorem{lemma}[theorem]{Lemma}
\newtheorem{claim}[theorem]{Claim}
\newtheorem{corollary}[theorem]{Corollary}
\theoremstyle{definition}
\newtheorem{definition}[theorem]{Definition}
\newtheorem{assumption}[theorem]{Assumption}
\theoremstyle{remark}
\newtheorem{remark}[theorem]{Remark}

\newcommand{\ignore}[1]{}

\newcommand{\ProblemName}[1]{\textsc{#1}}
\newcommand{\kzC}{\ProblemName{$(k, z)$-Clustering}\xspace}
\newcommand{\kMedian}{\ProblemName{$k$-Median}\xspace}
\newcommand{\kMeans}{\ProblemName{$k$-Means}\xspace}
\newcommand{\kMeanspp}{\ProblemName{$k$-Means++}\xspace}
\newcommand{\kCenter}{\ProblemName{$k$-Center}\xspace}

\newcommand{\RR}{{\mathbb R}}
\DeclareMathOperator{\E}{\mathbb{E}}

\DeclareMathOperator{\sdim}{sdim}
\newcommand{\dist}{\text{dist}}
\newcommand{\eps}{\varepsilon}
\newcommand{\opt}{\text{OPT}}
\newcommand{\cost}{\text{cost}}
\newcommand{\calS}{\mathcal{S}}
\DeclareMathOperator{\poly}{poly}
\DeclareMathOperator{\tw}{tw}
\DeclareMathOperator{\pw}{pw}
\DeclareMathOperator{\proj}{proj}
\DeclareMathOperator{\OPT}{OPT}
\DeclareMathOperator{\Diam}{Diam}
\newcommand{\cand}{\mathbb{S}}
\newcommand{\greedy}{\mathcal{C}}

\DeclareMathOperator{\NN}{NN}
\newcommand{\minn}[1]{\min\{{#1}\}}
\newcommand{\maxx}[1]{\max\{{#1}\}}
\providecommand{\set}[1]{{\{#1\}}}

\def\compactify{\itemsep=0pt \topsep=0pt \partopsep=0pt \parsep=0pt} 

\newcommand{\colnote}[3]{\textcolor{#1}{$\ll$\textsf{#2}$\gg$\marginpar{\tiny\bf #3}}}
\newcommand{\xuan}[1]{\colnote{purple}{#1--Xuan}{XW}}
\newcommand{\rnote}[1]{\colnote{blue}{#1--Robi}{RK}}
\newcommand{\chris}[1]{\colnote{green}{#1--Chris}{CS}}

% Todonotes is useful during development; simply uncomment the next line
%    and comment out the line below the next line to turn off comments
%\usepackage[disable,textsize=tiny]{todonotes}
\usepackage{todonotes}


% The \icmltitle you define below is probably too long as a header.
% Therefore, a short form for the running title is supplied here:
\icmltitlerunning{Coresets for Fair $k$-Clustering}

\begin{document}

\twocolumn[
\icmltitle{Poly($k/\varepsilon$)-Sized Coresets for Fair $k$-Clustering and Other Cool Cardinality Constraints}

% It is OKAY to include author information, even for blind
% submissions: the style file will automatically remove it for you
% unless you've provided the [accepted] option to the icml2022
% package.

% List of affiliations: The first argument should be a (short)
% identifier you will use later to specify author affiliations
% Academic affiliations should list Department, University, City, Region, Country
% Industry affiliations should list Company, City, Region, Country

% You can specify symbols, otherwise they are numbered in order.
% Ideally, you should not use this facility. Affiliations will be numbered
% in order of appearance and this is the preferred way.
\icmlsetsymbol{equal}{*}

\begin{icmlauthorlist}
\icmlauthor{Firstname1 Lastname1}{equal,yyy}
\icmlauthor{Firstname2 Lastname2}{equal,yyy,comp}
\icmlauthor{Firstname3 Lastname3}{comp}
\icmlauthor{Firstname4 Lastname4}{sch}
\icmlauthor{Firstname5 Lastname5}{yyy}
\icmlauthor{Firstname6 Lastname6}{sch,yyy,comp}
\icmlauthor{Firstname7 Lastname7}{comp}
%\icmlauthor{}{sch}
\icmlauthor{Firstname8 Lastname8}{sch}
\icmlauthor{Firstname8 Lastname8}{yyy,comp}
%\icmlauthor{}{sch}
%\icmlauthor{}{sch}
\end{icmlauthorlist}

\icmlaffiliation{yyy}{Department of XXX, University of YYY, Location, Country}
\icmlaffiliation{comp}{Company Name, Location, Country}
\icmlaffiliation{sch}{School of ZZZ, Institute of WWW, Location, Country}

\icmlcorrespondingauthor{Firstname1 Lastname1}{first1.last1@xxx.edu}
\icmlcorrespondingauthor{Firstname2 Lastname2}{first2.last2@www.uk}

% You may provide any keywords that you
% find helpful for describing your paper; these are used to populate
% the "keywords" metadata in the PDF but will not be shown in the document
\icmlkeywords{Clustering, Fairness, Coresets}

\vskip 0.3in
]

% this must go after the closing bracket ] following \twocolumn[ ...

% This command actually creates the footnote in the first column
% listing the affiliations and the copyright notice.
% The command takes one argument, which is text to display at the start of the footnote.
% The \icmlEqualContribution command is standard text for equal contribution.
% Remove it (just {}) if you do not need this facility.

%\printAffiliationsAndNotice{}  % leave blank if no need to mention equal contribution
\printAffiliationsAndNotice{\icmlEqualContribution} % otherwise use the standard text.

\begin{abstract}
TODO\\
\end{abstract}

%\section{Introduction}
\chris{The technical sections are unusable. Sometimes I may believe the claims, but the proofs are uniformly not rigorous. The exposition is also lacking, and the arguments are not complete. It is so bad, I think we need to start from scratch.}
Center-based clustering is well understood, both regarding the computational challenge of computing a good clustering, as well as the question of compressing a data set.
Contrary to this, clustering with capacity constraints is a notoriously difficult problem. An illustrative example is the (uniform) capacitated $k$-means problem: Given a set of points in $\mathds{R}^d$, the problem of computing a partition of the points into $k$ \textit{clusters} of equal size so as to minimize the sum of the squared distances from each point  to the mean of its cluster.
Such capacity constraints have recently received substantial attention as they frequently arise in important applications such as disparity-of-impact-based fairness.
The difficulty of finding a good clustering increases dramatically when adding capacity constraints. In the non-capacitated version, there exist a large number of constant factor approximation algorithms, often with very good running times. In the capacitated version, determining whether a polynomial time constant factor approximation algorithm exists is a major open problem. Similarly, constant factor approximations are only known for special cases of fair clustering.

A similar, though not as pronounced, dichotomy exists for computing coresets for these problems. Roughly speaking, a coreset $D$ of a point set $P$ is a small weighted subset such that for any candidate solution the cost of the $D$ and $P$ are within a $(1\pm \varepsilon)$ factor for some precision parameter $\varepsilon>0$.
In Euclidean spaces, coresets of size $\tilde{O}(k\cdot \varepsilon^{-4})$ exist for unconstrained $k$-means \cite{stoc}. 
Contrary to this, the state of the art coresets for fair clustering with a constant number of groups in Euclidean spaces are $\tilde{O}(k^7\varepsilon^{-5}\log n^6 (\log n + d))$ by \cite{BandyapadhyayFS21} and $O(k^3 \varepsilon^{-d-1})$ by \cite{HuangJV19}, where $n$ is the number of input points and $d$ is the ambient dimension of the point set. Thus, we have large dependencies on parameters that do not even appear for the unconstrained case.

In this paper we show that such a large gap between unconstrained clustering problems and a large number of clustering objectives is not necessary. Specifically, we show that for \emph{any} capacity constraint, a small coreset exists for both $k$-median and $k$-means. Thus we either improve on or give the first coreset construction for the following problems:
\begin{description}
    \item[Capacitated Clustering.] We give a coreset of size $\tilde{O}(k^2 \varepsilon^{-O(1)})$. This improves over the previous state of the art by \cite{cohen2019fixed}, who gave a corset of size $\tilde{O}(k^2\varepsilon^{-O(1)} \log^2 n )$.
    \item[Fair Clustering.] We are given a point set an a coloring of the input nodes. Each cluster is required to have the same ratio of colors as the entire point set. If the maximum number of different color combinations is bounded by $\Gamma$, our coreset has size $\tilde{O}(\Gamma k^2 \varepsilon^{-O(1)})$. This improves over the aforementioned previous results by \cite{BandyapadhyayFS21} and \cite{HuangJV19}. In particular, ours is the first coreset construction independent of both $n$ and $d$.
    \item[Clustering with Anonymity.] The size of every non-empty cluster is lower-bounded. This prevents clusters to consist of only only few points, which is motivated by privacy concerns. This problem was introduced by \cite{AggarwalPFTKKZ10}. Our coreset has size $\tilde{O}(k^2 \varepsilon^{-O(1)})$. To the best of our knowledge, this is the first coreset for this problem.
    \item[Clustering with Diversity] Similar to the fair clustering problem, we are given a coloring of the input nodes. In every cluster, we would like no more than a $1/\ell$-fraction of the points to belong to a single color, for some $\ell>0$. The problem was introduced by \cite{LiYZ10}. If the maximum number of distinct colors is bounded by $\Gamma$, our coreset has size $\tilde{O}(\Gamma k^2 \varepsilon^{-O(1)})$. To the best of our knowledge, this is the first coreset for this problem.
\end{description}

% This is rather unfortunate: clustering problems arise very naturally in a large number of fields and in particular in data analysis and unsupervised machine learning, and for several
% applications, there are constraints on the structure of the desired clusters. 
% For example, if each element of the clustering input belong to a population, one may 
% want to compute  \textit{fair} clustering, namely a clustering where for each cluster,  for each population, the percentage of individual of the population is the same in the cluster than for the entire input, hence providing representative clusters.
% This problem was introduced in the seminal work of \cite{NIPS2017_978fce5b} and has gained a lot of attention from machine learning community since then.
% Another example is the $\kappa$-anonymity framework where to ensure some level of privacy one may want to partition a given input into clusters of very similar elements and then act  on the centroids of the clusters rather than on the elements themselves. To achieve some  minimal amount of privacy (in fact anonymization), it is required that each cluster is  of some minimal size, say $\kappa$. 
% This can naturally be cast as a $k$-means problem with minimal-cluster-size (here $\kappa$). 
% Obtaining efficient solutions to the above problems has been at the heart of a variety of
% works and remains a major challenge.  \todo{Add biblio}

% One natural way to approach these problems is to reduce the input size as much as possible so as
% to enable more computationally expensive methods. For example, if 
% one aims at clustering the data in a few clusters (e.g.: $k = 10$), one may tolerate an 
% algorithm with running time $2^k$ poly($n$) but not $n^{k}$. It is thus desirable to find
% ways of reducing the input size. Central to these approaches
% is the notion of \textit{coreset}: A coreset is a small weighted subset of the input data set that can serve as a proxy for the optimization problem; In other words: a choice  
% of centers for the coreset is good choice of centers if and only if it is a good choice of centers for the original input.
% Despite the hardness of constrained clustering problems, it has been shown that small-size coresets for these problems do exist and can be computed efficiently\todo{add citation}.

\subsection{Related Work}
\paragraph{Coresets} Following a long line of work \cite{BachemL018,BecchettiBC0S19,BravermanJKW21,Chen09,stoc,FL11,FeldmanSS20,
FengKW21,FGSSS13,HaK07, HaM04, huang2020coresets, LS10, SohlerW18}, we now have coresets of size $\tilde{O}(k\varepsilon^{-4})$ for $k$-median and $k$-means in Euclidean spaces \cite{stoc}.
Coresets for clustering with capacity constraints is a comparatively recent topic, initially proposed by \cite{SSS19}, and then subsequently improved by \cite{BandyapadhyayFS21,HuangJV19}. Related questions of making fair clustering algorithms more scalable can be found in \cite{BIOSVW19}.
\paragraph{Fair Clustering}
Fair clustering under disparity of impact was first proposed in the seminal work by \cite{CKLV17}. Though no constant factor approximation algorithm is known in general, \cite{CKLV17} were able to obtain such a result if the number of protected classes was two and both classes had (nearly) equal size. Follow up work (see e.g.~\cite{BCN19,BGKKRSS19,RS18,SSS19}) has considered the question of whether these
guarantees can be extended for multiple protected classes.  \cite{BGKKRSS19} and \cite{BCN19}) obtained a constant factor approximation violating the fairness guarantee by small additive terms.
Under the same assumptions as \cite{CKLV17}, \cite{BohmFLMS21} obtained a constant factor approximation.
For other clustering problems using the same fairness notion, we refer to recent work by Ahmadian et al.~\cite{AhmadianE0M20,AhmadianEK0MMPV20}.

\paragraph{Clustering with Other Cardinality Constraints}
Aside from fair clustering, capacitated clustering has received most attention \cite{AdamczykBMM019,CyganHK12,DemirciL16,Li15,Li17}. The best general approximation factor for $k$-median and $k$-means is still $\log k$. Constant factor approximation are only known if one allows for an exponential running time in $k$ \cite{AdamczykBMM019,Cohen-AddadL19}, or allows either a violation of capacities or number of clusters \cite{DemirciL16,Li15,Li17}. For clustering with anonymity, \cite{Arutyunova021} gave a constant factor approximation. If the number of clusters are assumed to be constant, \cite{DingX15} gave a PTAS for both clustering with diversity and clustering with anonymity.


\subsection{Preliminaries and Problem Definitions}
Throughout this paper, we consider $d$-dimensional Euclidean spaces, i.e. the distance between two points $a$ and $b$ is defined as $\|a-b\| = \sqrt{\sum_{i=1}^d (a_i-b_i)^2}$. We use $[n]$ to the denote the positive integers from $1$ to $n$.
The most general problem we consider is the \kzC clustering problem, where, given a point set $X$, a set of centers $C$, a positive integer $z$, and an assignment $c:X\rightarrow C$, we have
$$\cost(X,C) := \sum_{p\in X} \|p-c(p)\|^z.$$
Special cases include $z=1$, which corresponds to Euclidean $k$-median, and $z=2$, which corresponds to Euclidean $k$-means.
For ease of presentation, we present most of our results for Euclidean \kMedian with several constraints, including the ones described above. Our algorithms and analysis can be extended to \kMeans and the more general \kzC problems. Details can be found in the supplementary material.
We focus on assignments resulting from cardinality constraints, i.e. we are also given a mapping $w:C\rightarrow [|X|]$ with $\sum_{c_i\in C} w_i = |X|$ and the mapping $c$ is the mapping minimizing the cost such that the number of points served by $C_i$ is $w_i$. To emphasize that the mapping depends on $w$, we write $\cost(X,w,C)$.

Our aim is to compute coresets for the following evaluation queries.
\paragraph{Coresets for Clustering with Cardinality Constraints}
Let $X\subseteq \mathbb{R}^d$ be a point set. Let $C\subseteq \mathbb{R}^d$ be a set of at most $k$ centers and let $w:C\rightarrow [|X|]$ be a set of cardinalities such that $\sum_{c_i\in C} w_i = |X|$. We say that a weighted subset $D\subseteq X$ is a coreset for clustering with cardinality constraints if for every $C$ and every $w$, we have
$$ |\cost(X,w,C) - \cost(D,w,C)| \leq \varepsilon\cdot \cost(X,w,C).$$

We note that computing coresets for the aforementioned problems of fair clustering, capacitated clutsering, clustering with anonymity, and clustering with diversity can be  reduced to computing coresets for clustering with cardinality constraints.
This can be seen as follows. For any constraint, there exists as feasible clustering with a mapping $w$ satisfying this constraint. Since the mapping $w$ imposes an additional constraint on the clustering, its cost cannot become smaller. At the same time, the cost cannot become larger either, as $w$ was feasible for the original constraint. Thus, the mapping $w$ corresponds to a clustering of identical cost.

For the special case of fair clustering, this argument changes in minor ways. In fair clustering, each point is given a set of colors corresponding to some protected attribute. The constraint imposed on every clustering is that for every pair of colors $\{RED, BLUE, GREEN,\ldots \}$, ever cluster $C_i$ satisfies $\frac{a_{RED}}{b_{BLUE}} \leq \frac{|C_i\cap RED|}{|C_i\cap BLUE|} \leq \frac{b_{RED}}{a_{BLUE}}$, for $a_{RED}\leq b_{RED}$ and $a_{BLUE}\leq b_{BLUE}$. Note that points can have multiple colors.
We first partition the points in groups such that points from every group have the same subset of colors. For each of these groups, we now compute a coreset for clustering with cardinality constraints. A proof of the validity of this reduction can be found in Theorem 4.2 of \cite{HuangJV19}.


% \paragraph{Capacitated \kMedian} is the \kMedian problem with capacity constraint on each center. Given a data set $X\subseteq \mathbb{R}^d$ and a constraint vector $w:[k]\rightarrow \mathbb{R}_{\geq 0} $ such that $\sum_{i=1}^k w(i)=|X|$, we aim to find a $k$-subset $\{c_1,...,c_k\}\subseteq \mathbb{R}^d$ to minimize the sum of connection cost $\sum_{x\in X} d(x,\pi(x))$ where $\pi:X\rightarrow C$ is an assignment function such that $c_i$ can serve at most $w(i)$ points from $X$, i.e, $|\pi^{-1}(c_i)|\leq w(i)$. We use $\mathrm{cost}(X,w,C)$ to denote the value of the optimal assignment. To simplify the notation, we also regard $w$ as a weight function on $C$ and use $\mathrm{cost}(X,C)$ to denote $\mathrm{cost}(X,w,C)$ when there is a weight function $w$ on $C$.

% \paragraph{Coresets for Capacitated \kMedian} A weighted subset $D\subseteq X$ is called an $\varepsilon$-coreset of $X$ for capacitated \kMedian if for every weighted $k$-center $C\subseteq \mathbb{R}^d$,  
% $$
% \mathrm{cost}(D,C)\in (1\pm\varepsilon)\cdot \mathrm{cost}(X,C).\footnote{Here we extend the cost function to allow weighted subset where the data points can be served by centers fractionally}
% $$

% \paragraph{Coresets for Fair \kMedian} In fair \kMedian, multiple groups of the data set $G_1,G_2,...,G_l\subseteq X$ are given in advance. A capacity matrix $M\in \mathbb{R}^{l\times k}$ specifies the exact number points of every group in each cluster. Namely, if $C_j$ denotes the $j$-th cluster, $|M_{ij}|=|G_i\cap C_j|$. We use $\cost(X,M,C)$ to denote the cost of optimal assignment that satisfies $M$.

% A weighted subset $D\subseteq X$ is called an $\varepsilon$-coreset for fair \kMedian if for every $k$-center $C\subseteq \mathbb{R}^d$, $$
% \cost(D,M,C)\in (1\pm \varepsilon)\cdot \cost(X,M,C).
% $$

% At first glance, fair \kMedian seems similar to capacitated \kMedian but is more complicated. Fortunately, \cite{NEURIPS2019_810dfbbe} suggested the following reduction from coresets for fair \kMedian to coresets for capacitated \kMedian.

% \begin{lemma}[Variant of Theorem 4.2 in \cite{NEURIPS2019_810dfbbe}]
% In a fair \kMedian instance $X$, let $\Gamma$ denote the number of distinct collections of groups that one point can belong to. Suppose there is an algorithm $\mathcal{A}$ that constructs a coreset of size $S=\mathrm{Poly}(k/\epsilon)$ for capacitated \kMedian on any subset. Then one can construct an $O(\Gamma S)$-sized coreset of $X$ for fair \kMedian by running algorithm $\mathcal{A}$ for $\Gamma$ times.
% \end{lemma}



\subsection{Our Results}

\begin{theorem}[Coresets for Capacitated \kMedian] Given the input data set $X\subseteq\mathbb{R}^d$, there is an algorithm that constructs an $\varepsilon$-coreset of size for capacitated \kMedian with size $\tilde{O}(k^2/\epsilon^5)$ in linear time.
\end{theorem}

\begin{theorem}[Coresets for Fair \kMedian] Given the input data set $X\subseteq \mathbb{R}^d$ and multiple groups $G_1,...,G_l\subseteq X$. Let $\Gamma$ denote the number of distinct collections of groups that a single point $x\in X$ can belong to. Then there is an algorithm that constructs an $\varepsilon$-coreset of $X$ for fair \kMedian with size $\tilde{O}(\Gamma k^2/\epsilon^5)$.

\end{theorem}

\subsection{Technical Overview} \label{sec:tech}
The state of the art coreset algorithms for unconstrained clustering sample points proportionate to their sensitivity scores, which, informally, corresponds to the maximum contribution of a point in any given clustering. 
The sampled points are then weighted inversely proportionate to these sensitivity scores.
When constructing coresets in high-dimensional Euclidean spaces, this is essentially the only technique available to us; geometric methods used in \cite{HuangJV19} and \cite{SSS19} invariably have an exponential dependency on the dimension.

Unfortunately, using sensitivity sampling is very difficult to use when dealing with cardinality constraints. To see this, consider that a point set obtained non-uniform sampling does not, with good probability, exactly preserve the number of points. To see why this causes an issue, suppose we are given a point set that naturally falls into two clusters of equal size. The two clusters have an arbitrarily high distance from each other. If we subsample these points and over-estimate the number of points in one cluster by a small amount, then the surplus must be assigned to the other cluster. This incurs as a cost the distance between the two clusters which is arbitrarily large and thus the subsample cannot be a coreset.

To overcome this issue, previous sampling-based algorithms \cite{BandyapadhyayFS21,Cohen-AddadL19} rely exclusively on uniform sampling. This is the only sampling distribution that always exactly preserves the number of points. 
Here, we compute an initial solution $A$. For each center of $A$, we draw rings of exponentially increasing radii. For every ring, we now sample uniformly. 
Unfortunately, this blows up the coreset size by the number of rings required to do this partitioning, which is at least $\log n$.

We sidestep this issue by only using the rings for the points close to their respective centers, i.e. the points costing no more than $\varepsilon^{O(1)} \cdot \Delta_i$, where $\Delta_i$ is the average cost of points of cluster $i$ in the initial solution $A$.
The points further away are dealt with separately, we now only focus on the points in these close rings.
The standard way to proceed is to show that (1) the variance of the sample is bounded and (2) to enumerate all candidate solutions. 

For (1), we observe that if the centers are roughly $\varepsilon^{-1}\cdot r$ away from any point in the ring, where $r$ is the radius of the ring, then it does not matter which points are being served by that center as the cost of these centers is always perserved up to a $(1\pm \varepsilon)$ factor via the triangle inequality. If the centers are at distance at most $\varepsilon^{-1}\cdot r$, then the variance of the sample is bounded.

For (2), we simply enumerate over all solutions from scratch. The number of close points that can serve as candidate centers are roughly of the order $\exp(d)$, hence the number of candidate clusterings using these centers is at most of the order $\exp(kd)$. To conclude, we also bound the number of different cardinality assignments $w$. Naively, this number is ${n\choose k}$, which incurs a dependency on $\log n$. To avoid this dependency, we further bound the number of clustering assignments by observing that any close center of cardinality less than $\varepsilon^2/k n$ does not contribute much to the cost, and the contributions in cost between centers of cardinality $\varepsilon^2/k n$  and $n$ can be efficientyl discritized in powers of $(1+\varepsilon/k)$. Overall, this leads to the a coreset of size $\varepsilon^{-3} \log(\exp(kd)$, which, using recent dimension reduction techniques for coresets \cite{stoc,huang2020coresets}, can be improved to roughly $\tilde{O}(k\varepsilon^{-5})$.

This leaves us with the points which we did not place into a ring, which we denote by the \emph{far} points. The far points are now simply sampled proportionate to their cost in $A$. This does not preserve the number of points, so we rescale the weights correspondingly in post processing.
The analysis of the sampling procedure for the far points now also distinguishes between two cases. If the cost of the points is similar to $A$, the analysis is analogous to that of the ring case. If the points cost significantly more than in $A$, we use exchange arguments with the points in the inner rings. Specifically, we show that we can swap the assignment of any far point with a point in the interior rings. If this results in the far points having a cost similar to that in $A$, we are done. If this is not the case, then this means that all of the points in interior rings must have a large cost. Since the number of points in interior rings outnumber the far points, we can now simply charge to the cost of these points.


% \chris{stop here}


% Our algorithm adapts the framework of Chen \cite{C09}. Chen's framework first finds a good enough approximation $A$ for the input data set. Then for each cluster, points are partitioned into $O(\log n)$ rings with exponentially increasing radii and coresets will be constructed by uniform sampling for each ring. Following this framework, \cite{cohen2019fixed} and \cite{BandyapadhyayFS21} show how to design coresets for capacitated and fair \kMedian, of size proportional to $\log n$.

% To avoid the dependence of $O(\log n)$, there are two technical challenges. First, in \cite{cohen2019fixed} and \cite{BandyapadhyayFS21}, even for a single ring,  coresets size are still are least $\Omega(\log n)$.  Secondly, there are $\Omega(\log n)$ many rings in all the above algorithms, thus a standard divide-and-union approach can not get rid of an $\log n$ factor. 

% To resolve the first challenge, we provide a novel analysis for the uniform sampling in the single ring case. Our analysis is based on a careful discretization of the centers and capacities. Finally, we apply the recent terminal embedding \cite{narayanan2019optimal} and iterative size reduction \cite{braverman2021coresets} techniques to get rid of the dependence on the dimension $d$. See Section \ref{sec:basic} for more details.

% To resolve the second challenge, we apply a novel modification in the framework. Instead of partitioning into $\Omega(\log n)$ rings, we only pay attention to the closest $O(\log (k/\varepsilon))$ rings. 
% \label{sec:tech}. For the remaining points, we observe two important features: 1. Their weights at most $\poly(\varepsilon/k)$ fraction of all points; 2. They are at least $\poly(k/\varepsilon)$ far to the clustering centers. Based one these features, we can prove that a standard importance sampling on these far points suffices.

\section{Preliminaries and Problem Definitions}

Throughout this paper, we consider $d$-dimensional Euclidean spaces, i.e. the distance between two points $a$ and $b$ is defined as $\|a-b\| = \sqrt{\sum_{i=1}^d (a_i-b_i)^2}$. We use $[n]$ to the denote the positive integers from $1$ to $n$.
The most general problem we consider is the \kzC clustering problem, where, given a point set $X$, a set of centers $S$, a positive integer $z$, and an assignment $s:X\rightarrow S$, we have
$$\cost(X,S) := \sum_{p\in X} \|p-s(p)\|^z.$$
Special cases include $z=1$, which corresponds to Euclidean $k$-median, and $z=2$, which corresponds to Euclidean $k$-means.
For ease of presentation, we present most of our results for Euclidean \kMedian with several constraints, including the ones described above. Our algorithms and analysis can be extended to \kMeans and the more general \kzC problems. Details can be found in the supplementary material.
We focus on assignments resulting from cardinality constraints, i.e. we are also given a mapping $w:C\rightarrow [|X|]$ with $\sum_{s_i\in S} w_i = |X|$ and the mapping $s$ is the mapping minimizing the cost such that the number of points served by $S_i$ is $w_i$. To emphasize that the mapping depends on $w$, we write $\cost(X,w,S)$.
The set containing all feasible solutions $(S,w)$ is denoted by $\mathcal{S}$.

Our aim is to compute coresets for the following evaluation queries.
\paragraph{Coresets for Clustering with Cardinality Constraints}
Let $X\subseteq \mathbb{R}^d$ be a point set. Let $S\subseteq \mathbb{R}^d$ be a set of at most $k$ centers and let $w:C\rightarrow [|X|]$ be a set of cardinalities such that $\sum_{s_i\in S} w_i = |X|$. We say that a weighted subset $D\subseteq X$ is a coreset for clustering with cardinality constraints if for every $S$ and every $w$, we have
$$ |\cost(X,w,S) - \cost(D,w,S)| \leq \varepsilon\cdot \cost(X,w,S).$$

We also need something like this:

\begin{definition}\label{def:centroid-set}
Let $X\subset \mathbb{R}^d$ be a set of points in $d$-dimensional Euclidean space and let $k$ and $z$ be two positive integers. Let $\eps > 0$ be a precision parameter. Given a set of centers $\greedy$, a set $\cand$ is an \emph{approximate centroid set} for $(k, z)$-clustering on $P$ if it satisfies the following property.

For every set of $k$ centers $\calS \subset \mathbb{R}^d$, there exists $\tilde \calS \in \cand^k$ and a bijection $b: \calS \rightarrow \tilde \calS$ such that for all points $p \in P$ and all $s\in \calS$ with $\cost(p, s) \leq \left(\frac{4z}{\eps}\right)^z \cost(p, \greedy)$, it holds
\[|\cost(p, s) - \cost(p, b(s))| \leq \varepsilon \left(\cost(p, s) + \cost(p, \greedy)\right).\] 
\end{definition}

This is slightly stronger than the definition from \cite{stoc}. I believe it is necessary to account for cardinalities.

\section{Algorithm}
To compute a coreset with cardinality constraints, we start with perform the following steps.
\begin{enumerate}
    \item Find a constant factor bicriteria approximation $A$\footnote{An $(\alpha,\beta)$ bicriteria approximation is a solution $A$ such that the cost of $A$ is at most a factor $\alpha$ times the cost of an optimal solution and the number of clusters in $A$ is at most $\beta\cdot k$}.
    \item For every cluster $C_i$ with center $c_i$ of $A$, let $\Delta_i:=\frac{\cost(C_i,A)}{|C_i|}$ denote the average cost. Define the ring $R_{i,j}$ to be all the points at distance $[2^j\cdot \Delta_i,2^{j+1}\cdot \Delta_i)$. 
    \item {\bf Near Points} We denote the set of points in $R_{i,j}$ with $j<2\log \varepsilon$ by $N_i$. We add $|C_i\cap N_{i}|$ copies of $c_i$ to the coreset. 
    \item {\bf Ring Points} For each $R_{i,j}$ with $\log \varepsilon \leq j <2\log \varepsilon^{-1}$, we sample a set $\Omega(R_{i,j})$ of $\delta = \tilde{O}(\varepsilon^{-3} k \cdot \max(d,\varepsilon^{-2}))$ points uniformly at random, weighted by $\frac{|R_{i,j}\cap C_i|}{\delta}$.
    \item {\bf Far Points} We define the set of far points in rings $R_{i,j}$ with $2\log \varepsilon^{-1}\leq j$ by $F_i$. We sample a set $\Omega(F_i)$ of $\delta = \tilde{O}(\varepsilon^{-3} k \cdot \max(d,\varepsilon^{-2}))$ points proportionate to their distance to $c_i$. Weigh each point inversely proportionate to the sampling probability. Finally, uniformly scale the weights up or down such that the sum of weights equals $|F_i|$. 
\end{enumerate}

Our aim is to prove the following result.
\begin{theorem}
For any set of of points $X\subset \mathbb{R}^d$, we can compute in time $...$ a coreset with cardinality constraints $D$ of size at most $...$
\end{theorem}


\section{Analysis of Near Points}

Try to formulate the appropriate lemma here. We would like to argue that for any solution, the following inequality holds.
\begin{eqnarray*}
\left\vert\sum_{p\in N_i} (\cost(p,s(p)) - \cost(c_i,s(p))) \right\vert \\
\leq \varepsilon \cdot (\cost(N_i,S) + \cost(C_i,S)).
\end{eqnarray*}

For the proof, you only have to use the triangle inequality (appropriately generalized if we want it to hold for $k$-means).


\section{Analysis of Ring Points} 

Outline:

\begin{itemize}
\item We would like to argue that for any fixed solution $S$ (i.e. fixed capacities and fixed set of centers) the cost is preserved up to a $(1\pm \varepsilon)$ factor with probability $1-\delta$ if we sample $\varepsilon^{-2-z}\log \delta^{-1}$ many points.
\item Enumerate all possible choices of centers. 
\item Enumerate all possible choices of clusters sizes. 
\item Explain how to eliminate dependencies on the dimension.
\end{itemize}

We want to prove
\begin{lemma}
Let $D$ be subset of $\delta$ points, each chosen uniformly at random with replacement from $R_{i,j}$ and weighted by $\frac{|R_{i,j}|}{\delta}$. Then if $\delta > \alpha\cdot \varepsilon^{-3}\cdot (kd + \log 1/\pi)$, we have for all candidate solutions $(S,w)\in \mathcal{S}$ with probability at least $1-\pi$
\begin{eqnarray*}
|\cost(R_{i,j},S,w)-\cost(D,S,w)|\\
\leq \varepsilon\cdot \left(\cost(R_{i,j},S,w) + \cost(R_{i,j},\{c_i\})\right)
\end{eqnarray*}
\end{lemma}


The following notion will probably be useful.

Let $\mathcal{S}$ be the set of all candidate solutions. Let $\mathcal{S}_{i,j}$ be the set of solutions defined as $\mathcal{S}_{i,j} = \{S'\subset S\in \mathcal{S}~|~\cost(s,c_i)\leq \varepsilon^{-O(1)}\cdot 2^j\cdot \Delta_i \wedge s\in S'\}$.

\paragraph{Dealing with a single fixed solution}

You want to prove an analogue (but perhaps slightly more general? need to check) of Lemma 13 from \cite{cohen2019fixed}. Essentially, we want to show that any fixed solution from $S_{i,j}$, the cost is preserved.
To do this, you will need the run of the mill Bernstein's inequality type arguments that you can find either in \cite{cohen2019fixed}, or equivalently from \cite{stoc}.

\begin{lemma}
Let $D$ be subset of $\delta$ points, each chosen uniformly at random with replacement from $R_{i,j}$ and weighted by $\frac{|R_{i,j}|}{\delta}$. Then if $\delta > \alpha\cdot \varepsilon^{-3}\cdot \log 1/\pi$, we have for any candidate solutions $(S,w)\in \mathcal{S}$ with probability at least $1-\pi$
\begin{eqnarray*}
|\cost(R_{i,j},S,w)-\cost(D,S,w)|\\
\leq \varepsilon\cdot \left(\cost(R_{i,j},S,w) + \cost(R_{i,j},\{c_i\})\right)
\end{eqnarray*}
\end{lemma}

\paragraph{Discretization of centers}
The main goal is to discretize the set of potential centers. Let $\mathcal{S}$ be the collection of solutions in our discretization. Here, we want to show that for \emph{any} possible solutions $S$ and \emph{all} points $p\in R_{i,j}$ and $s\in S$, there exists a solution $S'\in \mathcal{S}$ and a bijection $b:S\rightarrow S'$ such that if $\|q-s\| \leq \varepsilon^{-1} \cdot \|q-c_i\|$ for some $q\in R_{i,j}$ then
$$ \left\vert \|p-s\| - \|p-b(s)\| \right\vert\leq \varepsilon\cdot \|p-c_i\|.  $$ 
You do this by casting an $\varepsilon^{O(1)}\cdot 2^j\cdot \Delta_i$-net of the ball centered around $c_i$ with radius roughly (up to constant factors) $\varepsilon^{-1} 2^j\cdot \Delta_i$.

Key Lemmas:


\begin{lemma}[Find this in literature, don't have to prove it yourself]
Let $B$ be the unit ball in $d$-dimensional Euclidean space.
There exists an $\varepsilon$-net of size $\left(1+\frac{2}{\varepsilon}\right)^{-d}$
\end{lemma}


\begin{lemma}
Let $R_{i,j}$ be the set of points at distance $[2^j\cdot \Delta_i,2^{j+1}\cdot \Delta_i)$ from $c_i$. Let $S$ be an arbitrary solution. Then there exists an approximate centroid set for $R_{i,j}$ of size $\varepsilon^{-\alpha\cdot kd}$, where $\alpha$ is an absolute constant.
\end{lemma}

\paragraph{Discretization of cardinalities}
Here, we need two arguments. First, we argue that the centers that are at distance at least $\varepsilon^{O(z)}\cdot 2^j\cdot \Delta_i$, we preserve the cost anyway.
For the centers that are close, we argue as follows. 
\begin{enumerate}
\item Centers with small cardinalities (less than $\varepsilon^{O(1)}\cdot \frac{|R_{i,j}|}{k}$ contribute only a negligible amount to the cost).
\item Define an exponential sequence to the base of $(1+\beta)$, where $\beta = \varepsilon^{O(1)}$ (just $\varepsilon$ might be possible, the calculation will show).
\item Let $Card:=\{\varepsilon^{O(1)}\cdot \frac{|R_{i,j}|}{k} \cdot (1+\beta)^t\|\}$ be the set of candidate cardinalities.
\item Bound the number of distinct cardinalities (should be of the order $\beta^{-1}\log k/\varepsilon$.
\item For a solution $S$ with cardinalities $w: S\rightarrow [|R_{i,j}|]$, we replacing the cardinalities with $w(s_i)$ with the largest cardinality in $Card$ that is smaller than $w(s_i)$. Let $\widehat{|R_{i,j}|}$ be the sum of the replacement cardinalities.
\item Argue that a minimum cost assignment of any subset of $\widehat{|R_{i,j}|}$ points of $R_{i,j}$ to the new cardinalities preserves the cost (up to an appropriate term).
\item Argue that the surplus $|R_{i,j}|-\widehat{|R_{i,j}|}$ can be assigned arbitrarily without it affecting the cost.
\end{enumerate}

Key Lemma:

Define the set of integers.
$Card:=\{\varepsilon^{O(1)}\cdot \frac{|R_{i,j}|}{k} \cdot (1+\beta)^t\|\}$.

First, prove that approximating all solutions with weights from $Card$ (when necessary) is sufficient.
\begin{lemma}
Let $\mathcal{S}_{i,j,H}= \{S_{i,j}\times H~|~s\in S\cost(s,c_i)\leq \varepsilon^{O(1)}\cdot 2^j\cdot \Delta_i\}$ and let $\mathcal{S}_{i,j,F} =\{S_{i,j}\times \mathbb{N}~|~s\in S\cost(s,c_i)\geq \varepsilon^{O(1)}\cdot 2^j\cdot \Delta_i\} $. Define $\mathcal{S'}:=\mathcal{S}_{i,j,H}\times \mathcal{S}_{i,j,F}$
If for every solution $(S',w')\in \mathcal{S'}$
$$ |cost(R_{i,j},w',S') - \cost(D,w',S')|\leq \varepsilon\cdot \cost(R_{i,j},w,\calS),$$ then for every solution $(S,w)\in \mathcal{S}$ 
$$|cost(R_{i,j},w',S') - \cost(D,w',S')|\leq \varepsilon\cdot \cost(R_{i,j},w,\calS).$$
\end{lemma}

Next, prove that approximating all solution solutions with weights can be done with few samples.

\begin{lemma}
Let $D$ be subset of $\delta$ points, each chosen uniformly at random with replacement from $R_{i,j}$ and weighted by $\frac{|R_{i,j}|}{\delta}$. Then if $\delta > \alpha\cdot \varepsilon^{-3}\cdot (kd + \log 1/\pi)$, we have for all candidate solutions $(S,w)\in \mathcal{S}_{i,j,H}$ with probability at least $1-\pi$
\begin{eqnarray*}
|\cost(R_{i,j},S,w)-\cost(D,S,w)|\\
\leq \varepsilon\cdot \left(\cost(R_{i,j},S,w) + \cost(R_{i,j},\{c_i\})\right)
\end{eqnarray*}
\end{lemma}

Finally conclude with a proof of the main lemma from this section.

\paragraph{Removal of $d$}

For $k$-means one can first use PCA. For the others, we have to apply terminal embeddings recursively. Details on how to do this are in \cite{stoc} and \cite{braverman2021coresets}.
%\section{General Case}
\chris{There is no exposition that ties the lemmas together. I have no idea how anyone is supposed to understand this. Also, since most of the proofs are in the appendix, it is very hard to follow, because it is poorly referenced.}

In this section, we show how to construct coresets for capacitated clustering beyond the ring case. Let $c$ be a center point, $P$ be a (positively) weighted data set,  $W:=\sum_{p\in P} w(p)$ denote the total weight of $P$, and $\mu:=\cost(P,c)/W$ denote the average cost. By Section \ref{sec:basic}, we can construct additive coresets for close points by decomposing them into multiple rings with exponentially increasing radius.\chris{I think we should define $C_{close}$ and $C_{far}$ somewhere and not just in the proof of lemma 1.4 or the statement of lemma 2.1. I have difficulty keeping the definitions straight otherwise.}

\begin{lemma} \label{lemma:closepoints}
Let $P_{close}:=\{p\in P| \cost(p,c)\leq \frac{k}{\epsilon^2}\cdot \mu\}$. Then one can construct a weighted subset $D$ of $P_{close} $ such that for any solution $S, w(S)=W$, we have
$$
|\cost(D,S)-\cost(P,S)|\leq \epsilon \cost(P,S)+\epsilon\cost(P,c).
$$
\end{lemma}

We remain to deal with far points in $P$. Let $P_{far}:=P\setminus P_{close}$. We employ the following sensitivty sampling to construct coresets for $P_{far}$.

\paragraph{SensitivitySampling} works as the following. 

\begin{itemize}
\item Take a sample $G$ of size $m=O(\frac{k}{\epsilon^3}\log \delta^{-1})$ from $P_{far}$ where every $p\in P_{far}$ is sampled with probability $\frac{\cost(p,c)}{\cost(P_{far},c)}$ and will be assigned weight $w_G(p):=\frac{\cost(P_{far},c)}{m\cdot \cost(p,c)}$. Here, $\delta$ is the failure probability that will be fixed later.
\item Let $D$ denote the coreset constructed by Lemma \ref{lemma:closepoints}. Let $D_{new}:=D\cap G$.
\item Scaling the weight of $D_{new}$ so as $w(D_{new})=W$. Return $D_{new}$ as an $\epsilon$-coreset for $P$.
\end{itemize}

\begin{lemma} \label{lemma:singlecluster}
$D_{new}$ is an $\epsilon$-coreset of $P$ for capacitated clustering.
\end{lemma}

We first prove our sample has coreset property for a single solution in Lemma \ref{lemma:singlesolution}. Then we can use a discretization technique and union bound to prove Lemma \ref{lemma:singlecluster}.

We call a $k$-set $S$ a \emph{regular $k$-set} if $|S|=k$ for each $s\in S$, $w(S)=W$ and $w(s)\geq \frac{\epsilon}{k}\cdot W$.
\begin{lemma} \label{lemma:singlesolution}
Let $S$ be a regular $k$-set. Then with probability $1-\delta$, 
$$
|\cost(P,S)-\cost(D_{new},S)|\leq \epsilon (\cost(P,c)+\cost(P,S)).
$$
\end{lemma}

In the following, we fix a regular $k$-set $S$. Consider an optimal assignment from $P$ to $S$ and for each $p\in P$, we use $S(p)$ to denote the center that serves $p$. For $s\in S$, naturally $S^{-1}(s)=\{p\in P|S(p)=s\}$. Let $P^{S,close}_{far}:=\{p\in P_{far}| \cost(p,S(p))\leq \epsilon^{-1}\cost(p,c)\}$ denote the set of points in $P_{far}$ that contribute at most $\epsilon^{-1}$ times their cost in the cluster centered at $c$. We prove the following lemmas.

\begin{lemma} \label{lemma:assignment}
Let $X$ denote a weighted subset of $P$ and for any $s\in S$, $w(S^{-1}(s))=w(X\cap S^{-1}(s))$, then
$\cost(X,S)=\sum_{x\in X} d(x,S(x))$. Namely, if $X$ has the same weight in all the clusters of $P$ generated by $S$, then an optimal assignment from $X$ to $P$ should connect each point to the same center as the assignment from $P$ to $S$.

\end{lemma}

\begin{proof}
For sake of contradiction, we assume $\cost(X,S)<\sum_{x\in X} d(x,S(x))$. Let $\delta>0$ be a small enough constant such that for each $x\in P$, $\delta \cdot w_X(x)\leq w_P(x)$. 

So $P$ contains a weighted subset $\delta\cdot X$. We observe that in the assignment from $P$ to $S$, each $x\in P$ is assigned to $S(p)$ by definition. However, as $\cost(X,S)<\sum_{x\in X} d(x,S(x))$ and $X$ has the same weight as $P$ in each cluster, there actually exists a better assignment of this subset $\delta X$ by using the same capacity of each center. By updating the assignment of $\delta X$, we obtain a new assignment from $P$ to $S$ with a smaller cost, which is contradict to the fact that $\cost(P,S)$ uses the optimal assignment.
\end{proof}


\begin{lemma} \label{lemma:close}
Let $G^{S,close}:=G\cap P_{far}^{S,close}$. Then with probability $1-\delta$,
\begin{eqnarray*}
&&\bigg{|}\sum_{p\in P_{far}^{S,close}}w(p)\cost(p,S(p))\\&-&\sum_{p\in G^{S,close}}w_G(p)\cost(p,S(p))\bigg{|}\\
&\leq& \epsilon\cdot \cost(P,c)+\epsilon\cdot \cost(P,S)
\end{eqnarray*}
\end{lemma}





\begin{lemma} \label{lemma:controlfar}
Let $H$ be a weighted subset of $P$. Suppose $w(H)\leq \frac{\epsilon^2}{k}\cdot W$, and for any $h\in H$, $d(h,S(h))
\leq d(h,c)/\epsilon$, then $$\sum_{h\in H} w(h)\cdot d(h,S(h))\leq O(\epsilon) \cdot (\cost(P,c)+\cost(P,S))$$.
\end{lemma}


The next lemma controls the total weight of $D_{new}$ before scaling.

\begin{lemma} \label{lemma:weight}
With probability at least $1-\delta$,
$$
|w_G(G)-w(P_{far})|\leq \epsilon\cdot W.
$$
\end{lemma}

\chris{the following claim is wrong, and not salvageable. I attempted to read the proof in the appendix. I am not sure what is supposedly proven, but it is bogus, lacking any semblance of rigor. any future proof attempt should avoid the ideas contained in the proof of the appendix}
\begin{lemma} \label{lemma:far}
øLet $P_{far}^{S,far}:=P_{far}\setminus P_{far}^{S,close}$ and $G^{S,far}:=G\cap P_{far}^{S,far}$, then 
\begin{itemize}
    \item $\sum_{p\in P_{far}^{S,far}} w(p)d(p,S(p)) \leq O(\epsilon)\cdot (\cost(P,S)+\cost(P,c))$
    \item  $\sum_{x\in G_{far}^{S,far}} w_G(x)d(x,S(x)) \leq O(\epsilon)\cdot (\cost(P,S)+\cost(P,c))$
\end{itemize}


\end{lemma}



So far, we can prove the following lemma that connecting each coreset point the same center as in the optimal assignment from $P$ to $S$ can yield a close assignment. 

\begin{lemma} \label{lemma:rawcoreset}
Let $D_{new}$ denote the coreset produced by \textsf{SensitivitySampling}. Then with probability $1-\delta$,
\begin{eqnarray*}
&&\bigg{|}
\sum_{x\in D} w_D(x)d(x,S(X))-\cost(P,S)
\bigg{|}\\
&\leq& O(\epsilon)\cdot (\cost(P,S)+\cost(P,c)).
\end{eqnarray*}
\end{lemma}

\begin{proof}
A combination of Lemma \ref{lemma:weight}, Lemma \ref{lemma:far}, Lemma \ref{lemma:close}.
\end{proof}

\chris{avoid "proofs" like these. given the lack of rigor everywhere else, I wouldn't trust it}

A remaining issue is that we need to control $\cost(D_{new},S)$ instead of $\sum_{x\in D_{new}} w_D(x)d(x,S)$. We need to show that the latter is close to the optimal assignment $\cost(D_{new},S)$.  We further make a slight modification on the surplus to apply Lemma \ref{lemma:assignment}.

\begin{lemma} \label{lemma:substitute}
There is a weighted subset $Q\subseteq P$ such that 

\begin{itemize}
    \item for each $s\in S$, $w_Q(S^{-1}(s)\cap Q)=w(S^{-1}(s))$, 
    \item $
|\cost(D_{new},S)-\cost(Q,S)|\leq O(\epsilon \cdot \cost(P,c)+\epsilon\cdot \cost(P,S)),
$
and 
\item connection cost on $D_{new}$ is close to that on $Q$, i.e,
\begin{eqnarray*}
&&\bigg{|}\sum_{x\in D_{new}} w_D(x) d(x,S(x))\\&-&\sum_{x\in Q} w_Q(x) d(x,S(x))
\bigg{|}\\&\leq& O(\epsilon \cdot \cost(P,c)+\epsilon\cdot \cost(P,S)).
\end{eqnarray*}
\end{itemize}
\end{lemma}

\begin{proof}
Recall that $D_{new}=D\cap G$, we oberseve that we can repalce $D$ with $P_{close}$ by paying at most $\epsilon (\cost(P,S)+\cost(P,c))$ by Lemma \ref{lemma:closepoints}.

It remains to deal with surplus. By Lemma \ref{lemma:weight}, we scale the weight only by a factor of $1\pm O(\epsilon)$. So each cluster's weight differs by an $O(\epsilon)$ fraction. Hence we are able to deal with the surplus, by paying at most $\epsilon\cdot  \cost(P,c)$, via transporting through the center $c$.
\end{proof}

We are ready to prove Lemma \ref{lemma:singlesolution}
\begin{proof}[Proof of Lemma \ref{lemma:singlesolution}]
A combination of Lemma \ref{lemma:substitute}, Lemma \ref{lemma:assignment}, and Lemma \ref{lemma:rawcoreset}.
\end{proof}



%\section{Experimental evaluations}
%some intro
While our algorithms work for a variety of capacity constraints, the most common implementations focus on fair clustering. Thus we focus on fair clustering in our experimental evaluation. 

\paragraph{Algorithms}
We used an implementation of our algorithm, as well as the algorithms studied in \cite{HuangJV19}. These algorithms are (1) their $k^3 \varepsilon^{-d-1}$ coreset algorithm, henceforth denoted by $HJV$, (2) uniform sampling, and (3) a variant of BICO \cite{FGSSS13} which was adapted for fair clustering by \cite{SSS19}, henceforth denoted BICO.

For our algorithm, we used $k$-means++ \cite{ArV07} to find an initial (bicriteria) approximation with $2k$ centers with 10 repetitions.
The sampling algorithm received a budget matching the calculated coreset for HJV, split  uniformly across all color groups. Each cluster and ring got a point budget proportional to the cluster and ring in the approximate solution. Budgets below 0 were rounded up to have representatives for each cluster.

In the algorithm we need to parameterize $\epsilon$ properly such that the ring sizes fit the wanted coreset size. We do this by constructing an initial coreset and using the emperic distortion to calculate the ring sizes for the coreset.  

\paragraph{Data Sets}
The datasets used are \textbf{Adult} \cite{adult_dataset}, \textbf{bank} \cite{MORO201422}, \textbf{Census1990} \cite{Census1990_dataset} by subsampling uniformly, \textbf{Diabetes} \cite{Diabetes_Dataset} with one or multiple sensitive features. For each of the datasets we use the numerical features.
\begin{table}[h]
\centering
\caption{Protected groups}
\begin{tabular}{|c|c|c|}
\hline
Dataset & protected features & groups\\
\hline
    Adult & Sex,Marital-status & 14 \\
    Bank & Marital, Default & 12 \\
    Census & Sex, Marital & 10 \\
    Diabetes & Gender, Age & 20 \\
\hline
\end{tabular}
\end{table}

Additionally, we also considered the normalized variants, initially proposed by \cite{HuangJV19}. Here, every feature is scaled by the entry of largest magnitude. This improves the performance of uniform sampling algorithm and thus we have a wider range of competing algorithms.

\paragraph{Setup}
The experiments are run on 4-core laptop CPU with 16GB ram. We implement our algorithm in Java and compare it to the Java implementation of HJV\cite{HuangJV19}, C++ implementation of BICO and Java implementation of uniform sampling. For solving LPs we use IBM CPLEX \cite{CPLEX}.

%What/how?
%The datasets used are \textit{Adult}\cite{adult_dataset}(d=6, $std_{min}=2.5$,$std_{max}=1.9\cdot10^5)$, \textit{bank}\cite{MORO201422}(d=10, $std_{min}=0.6$,$std_{max}=260)$, \textit{Census1990}\cite{Census1990_dataset}(d=13, $std_{min}=0.2$,$std_{max}=4)$, \textit{Diabetes}\cite{Diabetes_Dataset}(d=8, $std_{min}=0.9$,$std_{max}=10.7)$ with one or multiple sensitive features.

\paragraph{Aims and Method of Evaluation}
We seek to measure that the distortion introduced by our coreset for several solutions is competitive compared to HJV. Unfortunately, measuring the distortion analytically is a challenging task, so we use heuristics to get a lower bound on the distortion for every algorithm.

The experimental setup is based on the code by \cite{HuangJV19}, where we evaluate our importance sampling algorithm in the same manner as they evaluate the uniform sampling algorithm.

Following the definition of \cite{HuangJV19} we have a fair clustering of l features if $C=\{C_1...C_k\}$ is a k-partitioning of the dataset $X$, satisfying that $|C_i\cap P_j|=F_{ij}$ $\forall i\in [k], j\in [l]$. 

We draw $500$ independent random samples of (F,C), F is the assignment constraint for each of the groups, and C being random candidate centers. C is picked uniformly among the points of the original dataset and F is a random partition of the colors across centers. The fair assignment problem is formulated as a LP as in \cite{HuangJV19}. We calculate the distortion as   $|\frac{Cost(S,F,C)}{Cost(X,F,C)}-1|$ and report the maximum. We explore results on both normalized and original data, the latter of which can be seen in the appendix. 

We report the distortion of the coresets, the construction time of our sampling method and HJV, the average time to evaluate the objective on the original data and the coreset by our algorithm. We omit construction time of BICO and uniform as they have already been explored in \cite{HuangJV19}.  
\paragraph{Results}
Table \ref{tab:result_means} summarize the results of the experimental evaluation. We see that for all datasets the sampling algorithm matches or improves the distortion by HJV and BICO. For larger coreset sizes our algorithm outperforms uniform-sampling, whereas the performance for small coresets are comparable in some cases such as on the adult and diabetes datasets. 

Compared to BICO and HJV, our algorithm generally performs better on normalized data sets, despite the data not being particularly high dimensional. Uniform sampling performs very poorly compared to all other methods on Adult and Bank, but remains competitive on Diabetes and Census.
Uniform sampling performs competitively with our solution on smaller coreset sizes. In these cases, we believe it is more important to draw points from every clustering. Since the clusterings we construct have similar size, our sampling does not yet have an opportunity to distinguish itself from uniform sampling in a meaningful way. 

Since our coresets are always constructed in linear time, the running time of our algorithm does not depend particularly on the size of designated coreset. Contrarywise, HJV exhibits a significantly higher dependency on the input size, making it less scalable.



\begin{table*}[h]
    \centering
    \caption{Maximal distortions for each coreset, construction time and time to do the fair assignment step}
    \begin{tabular}{@{}c c|c|c|c|c|c|c|c|c}
     & Size & Our & HJV & BICO & Uni & $T_{Our}$ & T$_{HJV}$ & $T_{Asssign-OPT}$ & $T_{Assign-Our}$\\
    \hline
    \multirow{9}*{\rotatebox{90}{Adult}}
      &     15942 &   0.72\% &   3.22\% &   1.47\% &  19.81\% &      1137 &     20933 &   1526.3 &      366.3 \\
      &      9268 &   1.85\% &   6.54\% &   3.87\% &   5.13\% &      1219 &      9511 &        - &      139.8 \\
      &      4051 &   2.23\% &  10.32\% &   7.87\% &   2.89\% &      1233 &      6788 &        - &       68.7 \\
      &      2311 &   3.57\% &  13.35\% &  14.77\% &   3.21\% &      1218 &      3212 &        - &       33.3 \\
      &      1111 &   4.30\% &  17.80\% &  22.69\% &   4.92\% &       989 &      2327 &        - &       21.8 \\
      &       819 &   7.15\% &  19.59\% &  24.68\% &   5.05\% &      1191 &      1456 &        - &        7.8 \\
      &       728 &  10.92\% &  19.68\% &  24.38\% &   8.73\% &       895 &      1386 &        - &        6.1 \\
      &       607 &  13.55\% &  30.13\% &  28.47\% &   7.08\% &      1236 &      1333 &        - &        5.8 \\
      &       486 &  12.22\% &  35.51\% &  34.65\% &   6.36\% &      1056 &      1338 &        - &        4.1 \\
      \hline
  
    \multirow{7}*{\rotatebox{90}{Diabetes}}
      &      6514 &   2.26\% &  12.44\% &  12.64\% &   2.81\% &      3707 &     10028 &   4788.0 &       86.0 \\
      &      2819 &   3.04\% &  17.20\% &  21.12\% &   4.48\% &      3826 &      5289 &        - &       35.0 \\
      &      1390 &   2.50\% &  21.72\% &  28.33\% &   5.43\% &      7261 &      2726 &        - &       18.3 \\
      &       831 &   6.83\% &  22.96\% &  32.69\% &   8.41\% &      4644 &      1969 &        - &        6.9 \\
      &       477 &  15.12\% &  26.50\% &  39.45\% &   7.85\% &      3843 &      1045 &        - &        3.8 \\
      &       461 &  17.63\% &  26.41\% &  41.37\% &   6.73\% &      4460 &      1117 &        - &        3.5 \\
      &       453 &  13.02\% &  26.32\% &  40.57\% &   9.23\% &      3770 &      1153 &        - &        3.6 \\
      \hline
    \multirow{9}*{\rotatebox{90}{Census}}
      &      5276 &   1.36\% &   4.63\% &   3.76\% &   3.16\% &      1790 &     19818 &   2428.3 &       80.1 \\
      &      2011 &   1.80\% &   9.05\% &   7.05\% &   2.93\% &      1851 &      6960 &        - &       33.3 \\
      &       845 &   4.00\% &  12.62\% &  17.66\% &   4.46\% &      1693 &      3985 &        - &        6.6 \\
      &       472 &   3.38\% &  20.01\% &  21.02\% &   9.91\% &      1862 &      2056 &        - &        3.5 \\
      &       275 &   8.68\% &  26.66\% &  28.52\% &  10.38\% &      1760 &       814 &        - &        2.5 \\
      &       235 &   6.39\% &  26.85\% &  28.76\% &  14.15\% &      1722 &       992 &        - &        2.0 \\
      &       181 &  11.59\% &  40.63\% &  32.01\% &  11.61\% &      1668 &       678 &        - &        2.3 \\
      &       161 &  15.04\% &  40.65\% &  34.15\% &  10.65\% &      2168 &       641 &        - &        1.7 \\
      &       149 &  13.95\% &  39.99\% &  35.72\% &  14.20\% &      2440 &       801 &        - &        1.7 \\
     \hline
    \multirow{9}*{\rotatebox{90}{Bank}}
      &       245 &  4.91\% &  11.44\% &  11.12\% &  42.16\% &       631 &       803 &   1114.7 &        2.7 \\
      &       205 &  3.45\% &  10.99\% &  12.45\% &  25.49\% &       422 &       724 &        - &        2.2 \\
      &       184 &  4.48\% &  12.03\% &  16.08\% &  49.18\% &       579 &       712 &        - &        2.3 \\
      &       176 &  5.82\% &  10.90\% &  16.51\% &  52.79\% &       636 &       485 &        - &        4.3 \\
      &       162 &  6.18\% &  11.02\% &  21.51\% &  65.09\% &       756 &       953 &        - &        2.2 \\
      &       143 &  6.26\% &  10.85\% &  17.86\% &  41.38\% &       576 &      1250 &        - &        2.6 \\
      &       158 &  5.21\% &  10.83\% &  29.61\% &  27.06\% &      1012 &      1026 &        - &        2.0 \\
      &       143 &  5.97\% &  10.77\% &  15.69\% &  55.09\% &       661 &       736 &        - &        2.0 \\
      &       142 &  5.62\% &  10.28\% &  18.74\% &  91.04\% &       814 &       838 &        - &        1.5 \\
     \hline
     \end{tabular}
    \label{tab:result_means}
\end{table*}
\clearpage
% In the unusual situation where you want a paper to appear in the
% references without citing it in the main text, use \nocite
%\nocite{langley00}

\bibliography{refs}
\bibliographystyle{icml2022}


%%%%%%%%%%%%%%%%%%%%%%%%%%%%%%%%%%%%%%%%%%%%%%%%%%%%%%%%%%%%%%%%%%%%%%%%%%%%%%%
%%%%%%%%%%%%%%%%%%%%%%%%%%%%%%%%%%%%%%%%%%%%%%%%%%%%%%%%%%%%%%%%%%%%%%%%%%%%%%%
% APPENDIX
%%%%%%%%%%%%%%%%%%%%%%%%%%%%%%%%%%%%%%%%%%%%%%%%%%%%%%%%%%%%%%%%%%%%%%%%%%%%%%%
%%%%%%%%%%%%%%%%%%%%%%%%%%%%%%%%%%%%%%%%%%%%%%%%%%%%%%%%%%%%%%%%%%%%%%%%%%%%%%%
%\newpage
%\appendix
%\onecolumn
%\section{Appendix}
We study Euclidean capacitated \kMeans and the more general capacitated \kzC problem. 

We allow both client points and centers to have positive weights. Let $X$ denote a client set and $C$ denote a center set. For a client point $x\in X$, the weight $w_X(x)$ denotes its demand (multiplicity) and for a center $c\in C$, the weight $w_C(c)$ denotes its capacity. When it is not ambiguous, we simply use $w(x)$ or $w(c)$ in short of $w_X(x)$ or $w_C(c)$.

For a (weighted) client set $X\subseteq \mathbb{R}^d$ and a (weighted) $k$-center $C\subseteq\mathbb{R}^d,|C|=k$, we define the capacitated \kMeans cost of $X$ on $C$ by using the following network flow model.

\paragraph{Cost Function} The flow network $G$ contains the following elements. Let $S$ and $T$ denote the source and sink and $X\cup C$ are other nodes in $G$. For each $x\in X$, there is an arc from $S$ to $x$ with capacity $w(x)$ and cost $0$ per unit. For each $c\in C$,  there is an arc from $c$ to $T$ with capacity $w(c)$ and cost $0$ per unit. For each pair $(x,c)\in X\times C$, there is an arc from $x$ to $x$ with capacity $\infty$ and cost $\|x-c\|_2^2$ per unit. We let $\cost(X,C)$ denote the cost of the min-cost max-flow of $G$.

We remark that the above cost function matches the natural definition of capacitated \kMeans when the total weight of $X$ and $C$ are equal, namely $w(X)=w(C)$.


\paragraph{Coreset} For $D\subseteq X$, $D$ is called an $\epsilon$-coreset of $X$ for capacitated \kMeans problem if
(i) $w(D)=w(X)$, and
    (ii) for every $C\subset \mathbb{R}^d$ such that $|C|=k$ and $w(C)=w(X)$, $$|\cost(X,C)-\cost(D,C)|\leq \epsilon \cost(X,C).$$


\ignore{
We study \kMedian with capacitated centers in Euclidean space. A capacitated center is a center $c$ with capacity $f$ which denotes the maximum number of clients that $c$ can serve. We have $(c,f)\in \mathbb{R}^d\times [n]$ where $n$ is the number of clients.

For a set of clients $P$ where $|P|=n$ and a set of $k$ capacitated centers $C=\{(c_1,f_1),\cdots,(c_k,f_k)\}\subseteq \mathbb{R}^d\times[n]$, a valid assignment of $P$ to $C$ is map $\phi:P\rightarrow [k]$ such that $|\phi^{-1}(i)|\leq f_i$ for any $i\in [k]$. The cost of the assignment is defined as
$\mathrm{cost}(P,C,\phi):=\sum_{p\in P} \|p-\phi(p)\|$. The cost of $C$ is defined as $$
\mathrm{cost}(P,C):=\min_{\phi} \mathrm{cost}(P,C,\phi),
$$
i.e. the minimum cost over all valid assignments.
\begin{definition}
An instance of Euclidean capaciated \kMedian clustering $(P,\mathbb{F})$ denote a set of clients $P\subseteq \mathbb{R}^d$ and a candidate set of capacitated center $\mathbb{F}\subseteq \mathbb{R}^d\times [n]$ where $n=|P|$. The objective of capaciated \kMedian is to find the $k$-set $C\subseteq \mathbb{F}$ that minimizes $\mathrm{cost}(P,C)$.
\end{definition}
}

\subsection{Preliminaries}
\begin{lemma}[Bernstein's Inequality]\label{lemma:Bernstein}
Let $X_1,...,X_m$ be $m$ independent random variables. Suppose $\sum_{i=1}^m E[X_i^2]=V$ and $\max_{i\in[m]} |X_i|\leq M$ almost surely. Then for every $t>0$,
$$
\mathbb{P}\bigg(\big|\sum_{i=1}^m X_i-\sum_{i=1}^m E[X_i]\big|>t\bigg)\leq \exp\bigg\{-\frac{t^2}{O(V+Mt)}\bigg\}
$$
\end{lemma}

\begin{lemma}[Generalized Triangle Inequality] \label{lemma:GTI}
Let $a,b\geq 0,\epsilon>0$ and $z\geq 1$ then $$
(a+b)^z\leq (1+\epsilon)^{z-1}a^z+(1+\frac{1}{\epsilon})^{z-1} b^z
$$ and $$
|a^z-b^z|\leq \epsilon \cdot a^z+\big(\frac{z+\epsilon}{\epsilon}\big)^{z-1}|a-b|^z.
$$

\end{lemma}

\subsection{Coresets for a Single Ring}
\label{Lemma:SingleRing}
We first show how to construct coresets when the data set is contained in a ring.
\begin{definition}
Let $o\in \mathbb{R}^d$ and $r>0$. We define $$
 \mathrm{Ring}(o,r):=\{x\in \mathbb{R}^d| r\leq \|x-o\|_2\leq 2r\}.
 $$ 
\end{definition}

We will need the following lemma.

\begin{lemma}[Generalization of Lemma 13 in \cite{cohen2019fixed}]
\label{lemma:VincentCoreset}
Let $X\subseteq \mathbb{R}^d$ denote a weighted client set. Assume $X\subseteq \mathrm{Ring}(o,r)$ and $C$ is a weighted $k$-center such that $C\subseteq \mathbb{R}^d, |C|=k$. Let $D$ denote a uniform sample of size $m=O(\epsilon^3\log \delta^{-1})$ (where every $x\in X$ is regarded as $w(x)$ many independent copies) such that each sample is weighted by $\frac{w(X)}{m}$. Then with probability $1-\delta$, $$
|\cost(D,C)-\cost(X,C)|\leq \epsilon r^2 \cdot w(X).
$$
\end{lemma}

We carefully construct a family $\mathcal{F}$ of (weighted) $k$-center such that if the coreset property of $D$ holds on every $C\in \mathcal{F}$ then the coreset property of $D$ holds for (weighted) $k$-center.


\paragraph{Constructing $\mathcal{F}$} Let $N$ denote an $\epsilon r$-net of the ball $B(o,\frac{r}{\epsilon})$ with $|N|\leq \epsilon^{-O(d)}$. Let $H=\{i\cdot  \frac{\epsilon^3}{k}\cdot w(X)|i=0,1,...,
\lfloor\frac{k}{\epsilon^3}\rfloor\}$ denote the set of multiplicity of $\frac{\epsilon^3}{k}$ that is at most $w(X)$. Let $N\times H$ denote the set of weighted points $x$ such that $x\in N$ and $w(x)\in H$.
We define $$
\mathcal{F}:=\{C|C\subseteq N\times H,|C|=k\}.
$$
Note that $|\mathcal{F}|\leq \epsilon^{-O(kd)}\cdot (\frac{k}{\epsilon^3})^k$.

\begin{lemma} \label{lemma:discrete}
If for every $C\in \mathcal{F}$, \begin{eqnarray}
|\cost(X,C)-\cost(D,C)|\leq \epsilon r\cdot w(X). \label{eqn:ring_assumption}
\end{eqnarray} Then for any $C\subseteq \mathbb{R}^d$ such that $|C|=k$ and $w(C)=w(X)$, we have $$|\cost(D,C)-\cost(X,C)|\leq \epsilon \big(r\cdot w(X)+ \cost(X,C)\big).$$
\end{lemma}

\begin{proof}
Fix a general $C\subseteq \mathbb{R}^d$ such that $|C|=k$ and $w(C)=w(X)$.
Let $\mathrm{C}_{close}=\{c\in C|\|c-o\|_2\leq \frac{r}{\epsilon}\}$ and $C_{far}=C\setminus C_{close}$. We first prove the following simple fact that implies $C_{far}$ is actually not important.

\begin{claim} \label{claim:ringfar}
For every subset $Y_1,Y_2\subseteq X$ such that $w(Y_1)=w(Y_2)=w(C_{far})$, we have 
$|\cost(Y_1,C_{far})-\cost(Y_2,C_{far})|\leq O(\epsilon)\cdot \cost(Y_1,C_{far})$.
\end{claim}

\begin{proof}
We note that for every $y_1,y_2\in X$ and $c\in C_{far}$, by generalized traingle inequality Lemma \ref{lemma:GTI},  \begin{eqnarray*}|d(y_1,c)^2-d(y_2,c)^2|
&\leq& 
\epsilon \cdot d(y_1,c)^2+O(\frac{1}{\epsilon})\cdot d(y_1,y_2)^2\\
&\leq& \epsilon \cdot d(y_1,c)^2+O(\frac{ r^2}{\epsilon})\\
&\leq & O(\epsilon) \cdot d(y_1,c)^2
\end{eqnarray*}
where we have applied $d(y_1,y_2)\leq d(y_1,o)+d(y_2,o)\leq 4r$ and $d(y_1,c)\geq d(o,c)-d(o,y_1)\geq \Omega(\frac{r}{\epsilon})$.

Interpret $Y_1,Y_2$ as measures on $X$ and match the density arbitrarily, we have
$$
|\cost(Y_1,C_{far})-\cost(Y_2,C_{far})|\leq \int_{Y_1} O(\epsilon)\cdot d(y,\pi(y))^2 dy\leq O(\epsilon)\cdot \cost(Y_1,C_{far})
$$
where $\mu(y)$ denotes the center that servers $y\in Y_1$ in the corresponding assignment of $\cost(Y_1,C_{far})$.
\end{proof}

\begin{claim} \label{claim:ring_close_far}
\begin{eqnarray} \label{eqn:ring_close_far_1}
|\cost(X,C)-\cost(X,C_{close})-\cost(X,C_{far})|\leq O(\epsilon)\cdot \cost(X,C_{far})
\end{eqnarray} and 
\begin{eqnarray} \label{eqn:ring_close_far_2}
|\cost(D,C)-\cost(D,C_{close})-\cost(D,C_{far})|\leq O(\epsilon)\cdot \cost(D,C_{far})
\end{eqnarray}
\end{claim}

\begin{proof}
By definition we have $\cost(X,C)\leq \cost(X,C_{far})+\cost(X,C_{close})$. 

On the other hand, let $X_{close}$ denote the (weighted) subset of $X$ that are served by $C_{close}$ in the corresponding assignment of $\cost(X,C)$ and let $X_{far}=X\setminus X_{close}$. Then we have $\cost(X,C_{close})\leq \cost(X_{close},C_{close})$. By Claim \ref{claim:ringfar}, we have $\cost(X,C_{far})\leq \cost(X_{far},C_{far})+ O(\epsilon) \cost(X,C_{far})$. Thus we have,
\begin{eqnarray*}
\cost(X,C_{close})+\cost(X,C_{far})&\leq& \cost(X_{close},C_{close})+\cost(X_{far},C_{far})+O(\epsilon) \cost(X,C_{far})\\
&\leq &\cost(X,C)+O(\epsilon)\cost(X,C_{far}).
\end{eqnarray*}
Thus (\ref{eqn:ring_close_far_1}) holds. In the same way, (\ref{eqn:ring_close_far_2}) holds.
\end{proof}

As $D$ is supported on $X$, by Claim \ref{claim:ringfar}, we know that \begin{eqnarray}
|\cost(X,C_{far})-\cost(D,C_{far})|\leq O(\epsilon)\cdot \cost(X,C_{far}) \label{eqn:ring_DX_far}
\end{eqnarray}
Now by Claim \ref{claim:ring_close_far} and Claim equation (\ref{eqn:ring_DX_far}), we only remain to prove $$
|\cost(X,C_{close})-\cost(D,C_{close})|\leq \epsilon r^2\cdot w(X).
$$

We first find a set $C'\subseteq \mathcal{F}$ to replace $C_{close}$. For each $c\in C_{close}$, as $c\in B(o,\frac{r}{\epsilon})$, we can find a net point $c'\in N$ such that $\|c-c'\|_2\leq \epsilon r$. Let $h\in H$ denote the largest multiplicity of $\frac{\epsilon w(X)}{k}$ that is not greater than $w(c)$. We add $(c',h)$ into $C'$ for each $c\in C_{close}$ to construct $C'$.

By (\ref{eqn:ring_assumption}) we have that $|\cost(X,C')-\cost(D,C')|\leq \epsilon r^2 w(X)$. Thus we only remain to bound $|\cost(X,C')-\cost(X,C_{close})|$ and $|\cost(D,C')-\cost(D,C_{close})|$.

\begin{claim}
$$
|\cost(X,C')-\cost(X,C_{close})|\leq O(\epsilon)\cdot (r^2 w(X)+\cost(X,C_{close})).
$$ and
$$
|\cost(D,C')-\cost(D,C_{close})|\leq O(\epsilon)\cdot (r^2 w(X)+\cost(D,C_{close}))
$$
\end{claim}
\begin{proof}
Just triangle inequality.
\end{proof}

Thus we conclude the proof of Lemma \ref{lemma:discrete}.

\end{proof}



\ignore{
\begin{proof}
We fix a valid $k$-center $C=\{(c_1,f_1),...,(c_k,f_k)\}\subseteq \mathbb{R}^d\times[W]$, i.e., $\sum_{i=1}^k f_i\geq W$. For each $c_i$, let $\eta_i$ denote the number of points that $c_i$ serves in the optimal assignment of $P$.

Let $I_{far}:=\{i|\mathrm{dist}(c_i,P)\geq \frac{20R}{\epsilon^2}\}$ denote the set of indices of centers that are $\frac{20R}{\epsilon^2}$ far from every point of $P$. Let $\eta_{far}:=\sum_{i\in I_{far}} \eta_i$ denote the number of points that far centers serve.

If $\eta_{far}\geq \epsilon W$ then $\mathrm{cost}(P,C)\geq \epsilon W\cdot \frac{20R}{\epsilon^2}=\frac{20WR}{\epsilon}$ and $I_{c}=[k]\setminus I_{far}$. We note that $D$ has the same capacity as $P$, and for every $x\in D$ and $p\in P$, $\|x-p\|_2\leq 4R$ as $D$ and $P$ are both in ring $(c,R)$, thus $|\mathrm{cost}(D,C)-\mathrm{cost}(P,C)|\leq 4WR\leq \epsilon\cdot \mathrm{cost}(P,C)$. 

If $\eta_{far}<\epsilon W$, we first show that by affording $O(\epsilon n R)$ error, we can assume $\eta_i=f_i$ for every $i\in I_{c}$. To see this, for every $p\in P$, $i\in I_{c}$, and $j\in I_{far}$, $$
d(p,c_i)\leq \frac{20R}{\epsilon^2}+4R\leq d(p,c_j)+4R.
$$

So if $I_{c}$ does not serve its full capacity, we can move points served by $I_{far}$ to $I_c$ by payting at most $4\epsilon W R$. So we can assume $\eta_i=f_i$ for each $i\in I_c$. Thus $\eta_{far}=W-\sum_{i\in I_c} f_i$.

To compare $\mathrm{cost}(D,C)$ with $\mathrm{cost}(P,C)$, we construct the following center $C'\subseteq \mathbb{F}$. For each $i\in I_c$, let $c_i'$ be such that $|c_i'-c_i|\leq \epsilon R$ and $f_i'$ be a multiple of $w_1$ such that $f_i'-f_i\in [0,w]$, by construction, $\mathbb{F}$ should contain at least one such $(c_i',f_i')$. We let $C':=\{(c_i',f_i')| i\in I_c\}\cup \{(c,\epsilon W)\}$.

As $D$ is an $\epsilon$-coreset of $(P,\mathbb{F})$, we have $|\mathrm{cost}(D,C')-\mathrm{cost}(P,C')|\leq \epsilon\cdot \mathrm{cost}(P,C')$. We write $\mathrm{cost}(P,C)=\mathrm{cost}_{c}(P)+\mathrm{cost}_{far}(P)$ where $\mathrm{cost}_{c}(P)$ and $\mathrm{cost}_{far}(P)$ are connection cost produced by centers in $I_c$ and $I_{far}$. Similarly we write $\mathrm{cost}(D,C)=\mathrm{cost}_c(D)+\mathrm{cost}_{far}(D)$. We then prove $|\mathrm{cost}_{far}(P)-\mathrm{cost}_{far}(D)|\leq O(\epsilon W R)$, $|\mathrm{cost}_{c}(P)-\mathrm{cost}_(P,C')|\leq  O(\epsilon W R)$, and 
$ |\mathrm{cost}_{c}(D)-\mathrm{cost}(D,C')|\leq  O(\epsilon W R)
$ to conclude.

To see $|\mathrm{cost}_{c}(P)-\mathrm{cost}_(P,C')|\leq  O(\epsilon W R)$, we note that $C'$ has at most $w\cdot k$ more capacity than $I_c$ and one more center $c$ with capacity $R$, the difference can be at most $kw\cdot \frac{20R}{\epsilon^2}+\epsilon W\cdot 2R\leq O(\epsilon W R)$. Similarly, we have $ |\mathrm{cost}_{c}(D)-\mathrm{cost}(D,C')|\leq  O(\epsilon W R).
$ 

Finally, as $I_{far}$ serves at most $\epsilon W$ points and every pair of points are within distance $4R$, we have $|\mathrm{cost}_{far}(P)-\mathrm{cost}_{far}(D)|\leq 4R\cdot \epsilon W\leq O(\epsilon W R)$.
\end{proof}
}

\begin{theorem}

For a client set $X$ such that $X$ is contained in a ring $\mathrm{Ring}(o,r)$, uniform sampling of size $\tilde{O}(k/\epsilon^5)$ yields an $\epsilon$-coreset with constant probability. 
\end{theorem}

\begin{proof}
By Lemma \ref{lemma:VincentCoreset} and $|\mathcal{F}|\leq \epsilon^{-O(kd)}\cdot (\frac{k}{\epsilon^3})^k$, with constant probability, the coreset property holds for every $C\in \mathcal{F}$ for a uniform sample of size $\tilde{O}(\frac{kd}{\epsilon^2})$. By terminal embedding \cite{narayanan2019optimal} and iterative size reduction \cite{braverman2021coresets}, we can trade in $d$ with a factor of $\tilde{O}(\log k/\epsilon^2)$. Thus a coreset of size $\tilde{O}(\frac{k}{\epsilon^5})$ can be obtained.
\end{proof}

\begin{proof} [Proof of Lemma \ref{lemma:close}]

We use Bernstein's inequality. Let $X_i$'s, $i=1,2,...,m$ denote independent variables corresponding to the samples. 

For the $i$-th sample, suppose it is $p\in P_{far}$, we let $X_i=\frac{\cost(P_{far},c)}{m\cdot \cost(p,c)}\cdot \cost(p,S(p))$ if $p\in P_{far}^{S,close}$ and $X_i=0$ otherwise.

So $$\sum_{i=1}^m X_i=\sum_{p\in G^{S,close}}w_G(p)\cost(p,S(p)) $$
and 

$$\mathbb{E}[\sum_{i=1}^m X_i]=\sum_{p\in P_{far}^{S,close}}w(p)\cost(p,S(p))$$.

We bound the $L_{\infty}$-norm by
$$
M:=\max_{p\in P_{far}^{S,close}} \frac{\cost(P_{far},c)}{m\cdot \cost(p,c)}\cdot \cost(p,S(p))\leq \frac{\cost(P,c)}{m\epsilon}
$$

We bound the second order moment by,
\begin{eqnarray*}
V&:=&m\sum_{x\in P_{far}^{S,close}} 
\bigg{(}\frac{\cost(P_{far},c)}{m\cost(p,c)}
\bigg{)}^2\cdot \cost(p,S(p))^2 \cdot \frac{\cost(p,c)}{\cost(P_{far},c)}\\
&\leq &\frac{1}{m\epsilon}\cost(P_{far},c)\sum_{x\in P_{far}^{S,close}} \cost(p,S(p))\\
&\leq &\frac{\cost(P,c)\cdot \cost(P,S)}{m\epsilon}
\end{eqnarray*}

It remains to plug in $M$, $V$, together with $t=\epsilon\cdot \cost(P,c)+\epsilon\cdot \cost(P,S)$ and $m\geq \frac{1}{\epsilon^3}\log\delta^{-1}$.


\end{proof}


\begin{proof}[Proof of Lemma \ref{lemma:controlfar}]

By triangle inequality, we have 
\begin{eqnarray*}
\sum_{h\in H} w(h) d(h,S(h))&\leq& \sum_{h\in H} w(h) (d(h,c)+d(c,S(h)))\\
&\leq & \sum_{h\in H} w(h) (\epsilon d(h,S(h))+d(c,S(h)))\\
&\leq & \epsilon \sum_{h\in H} w(h) d(h,S(h))+ \sum_{s\in S} w(H\cap S^{-1}(s))\cdot d(c,s)\\
&\leq & \epsilon \sum_{h\in H} w(h) d(h,S(h))+ \sum_{s\in S} O(\epsilon)\cdot w(S^{-1}(s))\cdot d(c,s)\\
\end{eqnarray*}

where for the last inequality, we have used $$
w(H\cap S^{-1}(s))\leq \frac{\epsilon^2}{k}\cdot W\leq O(\epsilon)\cdot w(S^{-1}(s)).
$$

It suffices to prove $$
\sum_{s\in S} w(S^{-1}(s)) \cdot d(c,s)\leq \cost(P,c)+\cost(P,S)
$$

We further note that,
\begin{eqnarray*}
\sum_{s\in S}w(S^{-1}(s))\cdot d(c,s)&=&\sum_{s\in S}\sum_{p\in S^{-1}(s)} d(c,s)\\
&\leq & \sum_{s\in S}\sum_{p\in S^{-1}(s)} (d(c,p)+d(p,S))\\
&=&\sum_{s\in S}(\cost(S^{-1}(s),c)+\cost(S^{-1}(s),s))\\
&=&\cost(P,c)+\cost(P,S)
\end{eqnarray*}

\end{proof}


\begin{proof}[Proof of Lemma \ref{lemma:weight}]
We use Bernstein's inequality Lemma \ref{lemma:Bernstein}.

Let $X_i$'s, $i=1,2,...,m$ denote an independent sample of \textsf{SensitivitySampling}.

We first compute the expectation,
$$
\mathbb{E}[w_G(G)]=m\cdot \sum_{p\in P_{far}}\frac{\cost(p,c)}{\cost(P_{far},c)}\cdot \frac{\cost(P_{far},c)}{m\cost(p,c)}=w(P_{far}).
$$

We control the $L_{\infty}$-norm by
$$
M:=\max_{p\in P_{far}} \leq \frac{\cost(P_{far},c)}{m\cost(p,c)}\leq \frac{W\mu}{m\cdot \epsilon^2/k\cdot \mu}=\frac{k}{m\epsilon^2}\cdot W.
$$

We bound the second order moment by
\begin{eqnarray*}
V&:=&m\sum_{p\in P_{far}} \bigg{(}\frac{\cost(P_{far},c)}{m\cost(p,c)}
\bigg{)}^2\cdot \frac{\cost(p,c)}{\cost(P_{far},c)}
\\
&=&\frac{1}{m}\sum_{p\in P_{far}}\frac{\cost(P_{far},c)}{\cost(p,c)}\\
&=&\frac{kW}{m\epsilon^2}\cdot w(P_{far})\\
&\leq & \frac{W^2}{m}
\end{eqnarray*}

It remains to plug in $t=\epsilon\cdot W$, $M=\frac{k}{m\epsilon^2}\cdot W$, $V=\frac{W^2}{m}$, and $m=\frac{\epsilon^3}{k}\cdot \log\delta^{-1}$.
\end{proof}


\begin{proof}[Proof of Lemma \ref{lemma:far}]
By Lemma \ref{lemma:controlfar}, it suffices to show that $w(P_{far}^{S,far})\leq O(\frac{\epsilon^2}{k})\cdot W$ and 
$w_G(G_{far}^{S,far})\leq O(\frac{\epsilon^2}{k})\cdot W$. The former holds trivially, as $w(P_{far})\leq O(\frac{\epsilon^2}{k})\cdot W$. 

For the latter, we will show $w_G(G)\leq O(\frac{\epsilon^2}{k})\cdot W$. By Bernstein's inequality, we are able to show, with probability at least $1-\delta$, $\cost(G,c)\leq O(\cost(P_{far},c))$. Since $G$ is a subset of $P_{far}$, we have  $\min_{x\in G} d(x,c)\geq \frac{k}{\epsilon^2}\cdot \mu$, where we recall that $\mu=\cost(P,c)/W$ is the average cost. Thus $$w_G(G)\leq \frac{O(\cost(P,c))}{\frac{k}{\epsilon^2}\cdot \mu}\leq  O(\frac{\epsilon^2}{k})\cdot W.$$
\end{proof}

\section{Further experiments}

\paragraph{K-Means}
Table \ref{tab:means} summarizes the results of the experiments run on the original datasets. The algorithm by HJV performs best on the Adult and Bank datasets compared to our algorithm. This is likely due to the large large difference in magnitude of the different features, making it easy to get an epsilon coreset if you only preserve the distance in the directions of the large feature. This is demonstrated by the fact that HJV has a higher distortion on the normalized datasets.
As expected our algorithm performs best on census, which also has the largest dimensionality. We also see that in the original datasets the importance sampling approach achieves a consistently lower distortion than uniform sampling.
%Coment on time, and update tables.

\begin{table}[ht]
    \centering
    \caption{Means results}
    \begin{tabular}{@{}c c|c|c|c|c|c|c|c|c}
     & Size & Ours & HJV & BICO & Uni & $T_{Our}$ & T$_{HJV}$ & Obj-time & Obj-time$_{Our}$\\
    \hline
    \multirow{9}*{\rotatebox{90}{Adult}}
          &       886 &  11.62\% &  0.34\% &   2.70\% &   8.88\% &      1481 &       782 &   1567.3 &        6.6 \\
      &       689 &   9.66\% &  0.62\% &  12.84\% &  16.31\% &      1390 &       500 &        - &        4.2 \\
      &       573 &   7.46\% &  0.67\% &   2.08\% &  44.87\% &      1282 &       551 &        - &        3.7 \\
      &       561 &   6.87\% &  1.29\% &   2.03\% &  16.04\% &      1193 &       546 &        - &        3.3 \\
      &       435 &   6.47\% &  1.11\% &   4.33\% &  15.33\% &      1088 &       491 &        - &        2.7 \\
      &       403 &   8.08\% &  1.36\% &   3.10\% &  18.31\% &      1098 &       493 &        - &        2.7 \\
      &       410 &  11.19\% &  2.14\% &   2.88\% &  32.75\% &      1133 &       470 &        - &        2.4 \\
      &       411 &   8.64\% &  2.47\% &   4.89\% &  16.47\% &      1143 &       537 &        - &        2.4 \\
      &       392 &   7.24\% &  2.40\% &   5.26\% &  23.73\% &      1132 &       533 &        - &        2.5\\
\hline
\multirow{7}*{\rotatebox{90}{Diabetes}}
      &      3349 &   2.13\% &  10.42\% &   9.78\% &   4.65\% &      5891 &      6348 &   6377.3 &       45.7 \\
      &      1383 &   6.62\% &  14.70\% &  14.36\% &  10.70\% &      5092 &      3238 &        - &       18.6 \\
      &       986 &   9.18\% &  13.48\% &  18.57\% &   7.49\% &      5222 &      3130 &        - &       12.5 \\
      &       891 &   5.82\% &  13.31\% &  19.85\% &  12.06\% &      5056 &      3591 &        - &        7.3 \\
      &       799 &  10.66\% &  12.99\% &  21.27\% &  10.71\% &      4580 &      3350 &        - &        6.1 \\
      &       732 &   9.34\% &  13.44\% &  20.57\% &   8.88\% &      4295 &      2972 &        - &        8.6 \\
      &       656 &  15.30\% &  13.11\% &  20.43\% &  15.90\% &      4474 &      2932 &        - &        5.0 \\

      \hline
    \multirow{9}*{\rotatebox{90}{Census}}
      &      6256 &   0.78\% &   5.13\% &   7.70\% &   3.92\% &      2117 &     26280 &   1871.4 &      110.7 \\
      &      2093 &   1.68\% &  10.08\% &   8.85\% &   5.00\% &      1998 &      8023 &        - &       38.2 \\
      &       795 &   3.46\% &  15.75\% &  15.32\% &  12.67\% &      2302 &      3796 &        - &        8.8 \\
      &       484 &   4.03\% &  23.80\% &  20.03\% &   9.33\% &      2213 &      2325 &        - &        4.2 \\
      &       298 &   6.57\% &  25.58\% &  25.74\% &  13.05\% &      1841 &      1584 &        - &        2.5 \\
      &       211 &   9.34\% &  29.38\% &  27.88\% &  19.91\% &      1851 &       967 &        - &        2.0 \\
      &       123 &  15.55\% &  39.73\% &  33.81\% &  20.11\% &      1643 &       691 &        - &        1.3 \\
      &       124 &  15.21\% &  39.87\% &  34.64\% &  14.39\% &      1559 &      1746 &        - &        1.2 \\
      &       134 &  15.69\% &  39.94\% &  34.32\% &  33.07\% &      1574 &       794 &        - &        1.4 \\
     \hline
    \multirow{9}*{\rotatebox{90}{Bank}}
         &       420 &  4.73\% &  3.17\% &  4.06\% &  23.13\% &      1093 &       487 &    842.9 &        2.6 \\
      &       331 &  7.24\% &  3.10\% &  4.06\% &  13.84\% &      1080 &       480 &        - &        2.2 \\
      &       286 &  6.96\% &  2.71\% &  5.24\% &  14.11\% &      1026 &       410 &        - &        2.1 \\
      &       253 &  7.72\% &  2.79\% &  6.87\% &  19.88\% &      1264 &       478 &        - &        1.8 \\
      &       222 &  6.88\% &  2.77\% &  4.82\% &  17.38\% &       973 &       491 &        - &        1.5 \\
      &       199 &  7.23\% &  2.67\% &  5.35\% &  39.40\% &      1035 &       472 &        - &        1.6 \\
      &       196 &  9.07\% &  2.27\% &  3.88\% &  56.23\% &       961 &       515 &        - &        1.4 \\
      &       194 &  8.71\% &  2.92\% &  4.68\% &  24.71\% &      1094 &       436 &        - &        1.4 \\
      &       171 &  7.40\% &  2.41\% &  5.20\% &  26.83\% &       992 &       524 &        - &        1.3 \\

      \hline
     \end{tabular}
    \label{tab:means}
\end{table}

\paragraph{K-median}
Running k-median we see a similar similar pattern of the datasets that we have a low distortion on. Again the performance of our algorithm is better than uniform sampling. Additionally our algorithm performs better on the Bank dataset on k-median compared to k-means. 

\begin{table*}[ht]
    \centering
    \caption{Median results}
    \begin{tabular}{@{}c c|c|c|c|c|c|c|c}
     & Size & Ours & HJV & Uni & $T_{Our}$ & T$_{HJV}$ & Obj-time & Obj-time$_{Our}$\\
    \hline
    \multirow{9}*{\rotatebox{90}{Adult}}
        &       271 &   7.20\% &  1.93\% &  14.30\% &      1545 &       472 &   1187.5 &        2.1 \\
      &       222 &   9.18\% &  2.93\% &  13.50\% &      1403 &       383 &        - &        1.7 \\
      &       216 &   6.97\% &  4.60\% &  18.38\% &      1548 &       381 &        - &        1.7 \\
      &       184 &   9.24\% &  4.09\% &  17.35\% &      1439 &       385 &        - &        1.5 \\
      &       181 &  10.29\% &  9.37\% &  19.28\% &      1383 &       370 &        - &        1.7 \\
      &       181 &  10.33\% &  5.57\% &  14.42\% &      1408 &       329 &        - &        1.6 \\
      &       151 &   7.65\% &  4.16\% &  15.85\% &      1257 &       377 &        - &        1.5 \\
      &       142 &   7.74\% &  6.44\% &  24.92\% &      1417 &       319 &        - &        1.4 \\
      &       105 &   6.95\% &  5.63\% &  14.12\% &      1378 &       401 &        - &        1.0 \\
      \hline
    \multirow{5}*{\rotatebox{90}{Diabetes}}
      &      6670 &  0.52\% &  3.15\% &   2.19\% &      4780 &      5383 &   6442.4 &      102.0 \\
      &      3431 &  1.07\% &  4.45\% &   1.24\% &      4904 &      4238 &        - &       46.5 \\
      &      2180 &  1.58\% &  4.74\% &   2.68\% &      5134 &      3859 &        - &       29.7 \\
      &      1671 &  3.13\% &  5.18\% &   3.78\% &      4401 &      3760 &        - &       21.6 \\
      &       929 &  2.61\% &  7.43\% &   2.27\% &      4665 &      3445 &        - &        6.9 \\
      \hline
    \multirow{9}*{\rotatebox{90}{Census}}
      &     11817 &  0.40\% &   1.19\% &   0.69\% &      2378 &     19012 &   2005.6 &      215.5 \\
      &      7330 &  0.46\% &   2.04\% &   4.09\% &      2445 &      9196 &        - &      119.4 \\
      &      4599 &  0.53\% &   3.28\% &   2.62\% &      2505 &      4984 &        - &       76.7 \\
      &      3204 &  0.89\% &   3.90\% &   3.27\% &      2382 &      3140 &        - &       55.2 \\
      &      1953 &  1.05\% &   5.53\% &   1.93\% &      2408 &      1733 &        - &       40.0 \\
      &      1286 &  1.76\% &   9.36\% &   4.05\% &      2460 &      1189 &        - &       31.5 \\
      &       904 &  2.27\% &   9.96\% &   4.92\% &      2457 &       921 &        - &       10.7 \\
      &       790 &  1.85\% &  10.97\% &   4.03\% &      2386 &       856 &        - &        8.4 \\
      &       456 &  2.68\% &  10.38\% &   4.00\% &      2952 &       671 &        - &        4.7 \\
     \hline
    \multirow{7}*{\rotatebox{90}{Bank}}
     &      2145 &  1.81\% &  1.21\% &   3.93\% &      1529 &      1039 &   1037.3 &       35.6 \\
      &      1087 &  2.33\% &  1.87\% &   3.52\% &      1335 &       609 &        - &       14.8 \\
      &       758 &  3.78\% &  3.10\% &   7.21\% &      1591 &       538 &        - &        6.0 \\
      &       665 &  2.06\% &  2.81\% &   5.46\% &      1459 &       554 &        - &        5.1 \\
      &       470 &  5.13\% &  4.70\% &   9.11\% &      1589 &       640 &        - &        3.5 \\
      &       468 &  4.22\% &  4.81\% &  10.00\% &      1333 &       576 &        - &        3.6 \\
      &       177 &  6.24\% &  9.04\% &  23.88\% &      1362 &       485 &        - &        1.6 \\
      \hline
     \end{tabular}
    \label{tab:median}
\end{table*}
\paragraph{Possible explanation for poor performance on Adult}
We also explore the fact that we might miss some low cost clusters when using the approximation. We find approximate solutions for k=k and for $k=6$ where m is the size of the largest coreset we used. We see that the ratio between the many-center and the few center solution is largest for Adult, indicating that we may miss some low cost center.
\begin{table}[ht]
    \centering
    \begin{tabular}{c|c|c}
    Dataset & m & $\frac{Cost(X, k)}{Cost(X,m)}$\\
    \hline
        Adult & 880& 6000 \\
        Diabetes & 50162& 267 \\
        Census &6352& 80\\
        Bank &  409 & 253
    \end{tabular}
    \caption{Caption}
    \label{tab:my_label}
\end{table}





\end{document}


% This document was modified from the file originally made available by
% Pat Langley and Andrea Danyluk for ICML-2K. This version was created
% by Iain Murray in 2018, and modified by Alexandre Bouchard in
% 2019 and 2021 and by Csaba Szepesvari, Gang Niu and Sivan Sabato in 2022. 
% Previous contributors include Dan Roy, Lise Getoor and Tobias
% Scheffer, which was slightly modified from the 2010 version by
% Thorsten Joachims & Johannes Fuernkranz, slightly modified from the
% 2009 version by Kiri Wagstaff and Sam Roweis's 2008 version, which is
% slightly modified from Prasad Tadepalli's 2007 version which is a
% lightly changed version of the previous year's version by Andrew
% Moore, which was in turn edited from those of Kristian Kersting and
% Codrina Lauth. Alex Smola contributed to the algorithmic style files.
